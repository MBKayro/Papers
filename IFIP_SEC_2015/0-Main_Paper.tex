%% Start Paper
\documentclass[English, runningheads,a4paper]{llncs}
%
\usepackage{amsmath}
\usepackage{amsfonts}
\usepackage{amssymb}
\usepackage{makeidx}
\usepackage{graphicx}
\usepackage{epstopdf}
\usepackage{booktabs} 
\usepackage{multirow}
%\usepackage[left=2cm,right=2cm,top=2cm,bottom=2cm]{geometry}
\usepackage{url}
%%% Autors %%%
\urldef{\mailsa}\path|mbakiri@cdta.dz,|
\urldef{\mailsb}\path|Christophe.Guyeux@femto-st.fr|    
\newcommand{\keywords}[1]{\par\addvspace\baselineskip
\noindent\keywordname\enspace\ignorespaces#1}


\begin{document}
\mainmatter  % start of an individual contribution

%%%%%%%%%%%%%%%%%%%%%%%%%%%%%%%%%%%%%%%%%%%%%%%%%%%%%%%%%%%%%%%%%%%
% Start Define the Title
% 1.0
\title{Survey on Hardware Implementation of Random Number Generators on FPGA: Theories and Experiments Analysis}

% 1.2
% a short form should be given in case it is too long for the running head
\titlerunning{Survey on Hardware Implementation of Random Number Generators}

% the name(s) of the author(s) follow(s) next
% 1.3
\author{Mohammed Bakiri\inst{1} \and Christophe Guyeux\inst{2}}

% (feature abused for this document to repeat the title also on left hand pages)
% 1.4 
%\authorrunning{Mohammed Bakiri\and Christophe Guyeux\ss er}

% the affiliations are given next
%1.5
\institute{Center for Development of Advanced Technologies, Baba Hassan, Alger, Algeria\and
FEMTO-ST Institute, UMR CNRS 6174, University of Franche-Comte, 25000, France.\\
\mailsa
\mailsb}

% 1.6 Creer le Titre
\maketitle
%%%%%%%%%%%%%%%%%%%%%%%%%%%%%%%%%%%%%%%%%%%%%%%%%%%%%%%%%%%%%%%%%%%%%%%%

\begin{abstract}
this paper introduce a survey on PRNG on FPGA.\dots

\keywords{Random Number Generator, PRNG, TRNG, Chaotic PRNG, Cryptography, Security, FPGA}
\end{abstract}

\section{Introduction}
\label{Introduction}
Accordingly, for cryptography applications, PRNGs must have the three characteristics. The first of these, the length of the seed used in the RNG should be large enough (say, ρ ≈ 2200 or more), so that an adversary cannot simply run through all possibilities, the Random number generators based on linear recurrences modulo 2 are among the fastest long-period generators currently available but very fast RNGs with huge period length can indeed be constructed this way. The second, the output sequences of random numbers should be statistically indistinguishable from random sequences. The third, random numbers should be unpredictable to an adversary with limited system resources. Should not waste memory (the state should be represented in no more than roughly log2 ρ bits of memory) and allow efficient jumping ahead in order to obtain multiple streams and sub-streams. But these properties do not suffice to imitate independent random numbers.

\noindent Chaos theory is well-studied in mathematics and nature phenomenal that widely exists in nonlinear systems [1, 2, 3]. It has many exceptional good properties such as the sensitivity to initial conditions and parameters, the pseudo-randomness, the topological transitivity, being irregular, non-periodicity, unpredictability, and ability to reciprocal synchronization [4]. However, in practice these solutions have also some important drawbacks and limitations, because the chaotic nature of generated series is non-ideal, due to the limited precision of arithmetic operations and quantization. As a result, instead of random number sequences we get pseudo-random or periodic series. In practice, the chaotic nature of the obtained series is non ideal, due to the finite precision of arithmetic and quantization. As a result we get pseudo-random or periodic series.

\section{Terminologies and Basic Recall}
\label{Terminologies and Basic Recall}
Some definitions and Symbols resume
\subsection{Definitions}
 
\subsubsection{Definition:} 
A random bit generator is a device or algorithm which outputs a sequence of statistically independent and unbiased binary digits

\section{Random Number Generator: Theories and Classification}
We can find many implementation of RNG in Software and Hardware but all can be classifies generally as PRNG, TRNG and  Hybrid,  but new concept has been introduce this last year's defined by parallel and chaotic generator.

%%% All Type of PRNG
\subsection{PRNG}
Some of the algorithms that generate pseudo random numbers are Blum Blum Shub, Inversive congruential generator, ISAAC (cipher), Lagged Fibonacci generator, Linear congruential generator, Linear feedback shift register, Mersenne twister, generalized feedback shift register (GFSR), twisted GFSR (TGFSR), Multiply-with-carry, Well-Equidistributed-Long-period Linear, Xorshift and Cellular Automata. As well as separately implementation of these structures, with use one or several of them, hybrid PRNGs can be designed. An important property of all these generators is that they are special cases of a general class of generators whose state evolves according to a (matrix) linear recurrence modulo 2 and the bits that form the output are also determined by a linear transformation modulo 2 applied to the state.

%%%% All Type of PRNG
\subsubsection{LFSR}

The initial value of the LFSR is called the seed, and by changing the seed we can change the sequence and because the operation of the register is deterministic, the stream of values produced by the register is completely determined by its current (or previous) state. Likewise, because the register has a finite number of possible states, it must eventually enter a repeating cycle. However, an LFSR with a well-chosen feedback function can produce a sequence of bits which appears random and which has a very long cycle. That feedback function is called a maximal length polynomial. The bits in the LFSR state which influence the input are called taps. This is called the feedback polynomial or characteristic polynomial. it is known that an LFSR with more taps produces a better sequence of random numbers. However, on an FPGA, adding more taps minimizes the number of LUT-based shift registers that can be utilized. For example, in 4 bit LFSR if the taps are at the 4th and 3rd bits (as shown), then the feedback polynomial is x4 +x3 +1. 

A nonlinear feedback shift registers NFSRs must be included in a key stream generator design to remove the linearity in the encrypted
\subsubsection{BBS}
1.\\
\underline{In 2012, Khushboo Sewak and all.} The main purpose of this paper is to study the FPGA implementation of two 16 bit PN sequence generator namely Linear Feedback Shift Register (LFSR) and Blum-Blum-Shub (BBS). The use of feedback shift register permits very fast generatio PN sequence whereas BBS method requires a number of time consuming arithmetic operation as it is based on quadratic Congruential equation.
\subsubsection{Mastrine Tewister}
\label{Mastrine Tewister}
2.\\
\underline{In 2013, Shengfei Wu and al} In this paper, a hardware architecture for the generation of parallel long-period random numbers using MT19937 method was proposed. Most hardware implementations of MT19937 are straightforward non-parallelized implementations of the original C-code. The Mersenne Twister Method, which is a pseudorandom number algorithm based on a matrix linear recurrence over F2, is developed by Makoto Matsumoto in 1997. In order to get the long period and good equidistribution, the Mersenne Twister is cascaded with a tempering transform to compensate for the reduced dimensionality of equidistribution, the temper is defined in the case of Mersenne Twister. We use dual-port BRAMs in FPGA for the implementation and 3 degrees parallelization will be introduced as an example.
\subsubsection{Cellular Automata}
%%%\section{Random Number Generator: Theories and Classification} 
%

%% Single N-Dimention CA:
In 2004 [9] Guan et al. proposed a one dimensional CA where the rule in each cell changes dynamically based upon the states of the cells within a new neighborhood of three cells . Dubbed “Self-Programmable Cellular Automata” (SPCA), the rules are switched between 90 and 165 or 150 and 105. These rules were selected because they can be easily implemented with XOR gates. Certain combinations of rules and neighborhoods were shown to produce maximal length sequences with good quality random numbers.

In 2006 Leonidas Kotoulas and al, The proposed 1-d CA is based on the real time clock sequence and used for stream cipher. The authors show that by using rules based computer times sequence as year, months to seconds can generate initial state and the length of CA cells. the initial state configuration and simultaneously the length of CA cells the product of all the above numbers, namely day, month, year, hour, min and seconds was calculated. The execution time was decided to be t = x(60 - x). The first rule arises from the product of minutes by seconds. The second rule is the number of minutes divided by the number of seconds multiplied by a constant.

Where In 2009 Ding Jun and al, implement an efficient PRNG based on the classical CA with 32 cells using rule 30 is reported and prove a high PRNG performance. 

In 2010 Ioana Dogaru and Radu Dogaru, create an automatic software tool based on Algebraic Normal Form (ANF) representation to generate an RTL code of an hybrid CA depending on ID rules. the results show by using ANF representation  $y = [K_{0} xor K_{1}(U_{1}) xor K_{2}(U_{2}) xor K_{3}(U_{3}) xor K_{4}(U_{1}*U_{2}) xor ... K_{7}(U_{1}*U_{2}*U_{3})] xor mask$ with ID=101 and 3 neighbor are identical to Matlab results.

%% Combined N-Dimention CA:
In 2003 Tkacik proposed a hardware random number generator implemented on a custom IC which combines the outputs of a CA with an LFSR. A hybrid 90/150 rules an 37 bit CA was combined with a 43-bit LFSR. This maximal length configuration combined 32 bits from the CA and LFSR to produce a maximal length RNG. It was found that the LFSR and CA must be clocked at different frequencies to create a sequence of numbers that can pass all the DIEHARD tests. Then, in 2012 Juan C.Cerda and al, notice in Tkcide [2003] the combination must be clocked at different frequency to pass all NIST test, and present another combination PRNG using HCA using 90/150 and LFSR to solve this problem. The trick is XORing the last bit of HCA with the last bit of LSFR to generate 1-bit per clock cycle, and they found the best combination for a high quality of PRNG is 37-bit LFSR with 16-bit CA. then 2012, they compare they preview work LFSR/HCA with a SPCA 2004 that use 90/156 rules, and they find that even SPCA fail in one statistic test but give a better throughput than the LFSR/HCA

In 2007 Petre Anghelescu and al, propose a combination of two logic combinational circuit of Hybrid HCA where PRNG and block cipher for an encryption system, and the first HCA-1 use two rules 90/150 as a real-time key stream generator and the second HCA-1 use 51/153/195 rules. To select witch rules will be used by the block cipher HCA-2, the PRNG or HCA-1 generate an encrypting rules to switch the rules and that provide each cell has is own rules. 

In 2013 Lakshman Raut and David H. K. Hoe, show us another stream cipher design and combination of CA and LFSR but they introduce NLFSR block based on A2U2 design to resist more for a various forms of cryptanalysis, such as correlation attacks and algebraic attacks. Where CA and NFSR are both has feedback for each other and use a LFSR counter as input random, then the key bit stream will pass to a mixer mechanism to increase the complexity for decryption. 

In 2014, Dogaru Ioana and Dogaru Radu. introduce a comparison o two implementation of HCA as PRNG to maximize efficiency/throughput, where the first is a basic 63-bit HCA and the second is a chain of HCA(2 HCA) and they demonstrate a high ration of frequency/ares and cryptography by using a chain of HCA instead of single HCA.

%%% All Type of TRNG
\subsection{TRNG}
\textit{Phase-Locked Loop PLL}
In 2006 Martin Simka and al, The authors demonstrate by combined more than one PLL even parallel or series, can increase significantly sensitivity on the jitter and the output-bit of the generator compared to the use of one PLL. The configuration of multiple PLL are based on input/output length, CVO frequency and MUL/DIV factors (KM/KD).

In 2008 Michal Varchola and al, To demonstrate advantages of this platform a TRNG based on internal PLL was implemented, but TRNGs based on other principles can be also tested. Great advantage of Actel Fusion FPGA is the possibility to create whole system with on-chip analog measurements. The basic principle behind the TRNG shown in Fig. 3 is to extract the randomness from the jitter of the clock signals synthesized in the embedded analog PLLs [2]. The jitter is detected by the sampling of a reference signal CLJ using a rationally related (clock) signal CLK synthesized in the on-chip analog PLLs with frequencies. It was observed that 16-bit data could be transmitted each 2 periods of CopreMP7 system clock by ModelSim simulation.Two configurations of TRNG have been tested operating at following bit-rates - 40 kbps and 1 Mbps. Slower 40 kbps TRNG pass the test very well but faster 1 Mbps does not

In 2011 Martin Simka and al, In the paper we analyze behavior of the Phase-Locked Loop (PLL) based TRNG in changing working environment. The frequency of signals synthesized by PLL may be naturally influenced by chip temperature. We show what impact the temperature has on the quality of generated random sequence of the PLL-based TRNG. When considering PLL as a source of randomness, three parameters have to be taken into account: 1) the size of generated clock jitter; 2) available ranges of frequency dividers n, m and k together with the VCO maximum and minimum frequency; 3) bandwidth of the PLL loop filter. We can conclude that lower bandwidth of the feedback PLL loop in the configuration B causes higher number of the critical samples. This can be explained by the fact that lower bandwidth decreases PLL output jitter and thus increases the tracking jitter.  

%%%%%%
\textit{Ring oxialltor RO}
In 2004 Paul Kohlbrenner and Kris Gaj, propose a TRNG based on two ring oscillators clocked by different clock generated by an internal PLL on FPGA, and with low area implement by only one CLB slice. the authors also extract the jitter of the 2 RO using a simpler and eliminate any correlation between successive bits.

In 2006 Drie Schellekens and al, a review on Ring oxialltor definition [6][7]

In 2009 Cristian Klein and al, This paper focuses on the design and implementation of a high-quality and high-throughput true-random number generator (TRNG) in FPGA. Our first attempt was to create a TRNG based on [3], which uses one RO to sample the output of the other RO. We appreciated this approach due to the fact that the whole stream before the post-processing phase is random (although it might be biased a little bit). A post-processing phase is required which consists in a resilience function. A single inverter allows us to create ROs which both even and odd number of latches. To make sure that the inverter does not add more delay the inverter and the first latch are mapped to the same CLB. we chose as the resilience function a simple XOR of 2r-bits. increases the output period of the ring oscillators, it also increases the amount of jitter.

In 2003 K.H.Tsoi and al , Two FPGA based implementations of random number generators intended for embedded cryptographic applications are presented. The first is a true random number generator (TRNG) which employs oscillator phase noise, and the second is a bit serial implementation of a Blum Blum Shub (BBS) pseudorandom number generator (PRNG). the TRNG, oscillator phase noise was used. our implementation uses a very high frequency clock (up to 400 MHz) and does not require a scrambler to achieve good random output. There are several factors which affect the randomness of the output [22]. The first situation is that the duty cycle of Fh may not be 50%. 
In this situation, Fr will have unequal probability of being zero or one. the only off-chip components being two resistors and a capacitor for the TRNG low frequency oscillator. PRNG is a bit serial arithmetic logic unit (ALU). The BBS PRNG performs three functions: seed validation, squaring and modulo operations. Seed validation is
performed once only during initialization, and after that, a squaring and modulo operation are performed each iteration to produce Xi, In total, each iteration of the PRNG requires $(4.5*n^{2} + n)$ clock cycles, where n is the size of the modules in bits.

In 2009 KnutWold and Chik How Tan, In this paper, we analyze the TRNG designed by Sunar et al. (2007) based on XOR of the outputs of several oscillator rings. We propose an enhanced TRNG with better randomness characteristics that does not require postprocessing and passes the statistical tests. We have shown by experiment that the frequencies of the equal length oscillator rings in the TRNG are not identical. The difference is due to the placement of the inverters in the FPGA and the resulting routing between the inverters. In this paper, we examine more closely the TRNG based on oscillator rings proposed by Sunar et al. [2]. We show that the TRNG described in [2] is not random without postprocessing our TRNG has no bias and, therefore, no need for complicated postprocessing. The TRNG consists of several equal length oscillator rings connected to an XOR tree. The output from the XOR tree is sampled by a D flip-flop, and the output signal of the D flip-flop is then postprocessed in order to increase the entropy and remove bias from the random signal. The entropy source of the TRNG is the jitter created by each oscillator ring. The main concern of the authors of [10] is that the XOR-tree and the sampling D flip-flop cannot handle the high number of transitions from the oscillator rings. we suggest an enhancement of the TRNG based on the oscillator rings in [2] by adding an extra D flip-flop after each ring. The frequency will increase with the decreasing number of inverters. The tendency is that the bias increases with the increasing number of rings. however equal length oscillator rings will have the same frequency is not true when the dispersion is decreasing with increasing number of inverters. for restart experiment, the first data bits should be omitted in order to have good quality of the random sequence.

IN 2007 Markus Dichtl and Jovan Dj.Golic, The main objective of this work is to evaluate and analyze the amount of true randomness produced by these oscillators. This is achieved by using the restart
approach, which consists in repeating the experiments from identical starting conditions. Fibonacci and Galois ring oscillators [8] (FIRO and GARO, respectively) are both defined as generalizations of a ring oscillator (RO). They consist of a number, r, of inverters connected in a cascade together with a number of XOR logic gates forming a feedback in an analogous way. the feedback polynomial should be chosen to have a form $f(x) = (1+x)h(x)$. To increase randomness and robustness, it is also proposed to use an
XOR combination of a FIRO and a GARO (FIGARO). compare them with a classical RO based inverters, show very clearly that the classical ROs need more than 5 μs until they reach an approximately stable value and the FIROs and GAROs achieve a more or less stable standard deviation of their output voltages already after about 50 ns. 

%%%%%
\textit{self-timed ring STR}

In 2013 Abdelkarim Cherkaoui and al, The proposed true random number generator (TRNG) exploits the jitter of events propagating in a self-timed ring (STR) to generate random bit sequences at a very high bit rate. the stochastic models are not feasible or at least not plausible, because they combine intrinsically pseudo randomness with true randomness. 
In [9], we showed for the first time that self-timed rings (STR) are a highly suitable source of entropy. Based on these observations, in [10] we proposed the first TRNG principle based on STRs. Self-timed rings (STR) are oscillators that can provide events which are evenly spaced in time and distributed over half an oscillation period of one ring stage. Contrary to inverter ring oscillators, several events can propagate simultaneously in STRs thanks to the asynchronous hand- shake protocol. The signals resulting from the STR outputs are synchronized and their mutual position does not change over time. In contrast, the ring oscillator output signals from [7] drift in time and generate pseudo-randomness.

%%%%
metastability:

%%% All Type of Chaotic
\subsection{Chaotic Rundom Number Generator}
%%%%%%%%%%%%%%
Logistic
Based on earlier study in 2012, Pawel Dabal and all show us a study of HW implementation in FPGA of different chaos system as logistic and Hénon mapping and Frequency depended negative resistor FNDR. Many recent research illustrate an optimization using new methods and algorithms to have a better chaotic and stable random and Area/Throutput results.

In 2014 Pawel Dabal and Ryszard Pelka, The authors propose a study of an fast Pipeline PRNG based on chaotic logistic map,and to solve the problem of the short cycle, distortion and correlation, the authors implement two version of PRNG based on simple logistic map equation using pipeline processing to have a a high operation frequency and the ability to generate sequence at different start point. The first is logLUT based on on LUT blocs and high-speed carry line of the FPGA and the second is Log DSP that use directly DSP of FPGA, however they add delays to ensure parallel sequence generation and a complex initial sequence used for a better NIST test results. As results, many configuration and tests based on delays and precision of PRNG applied to have the most combination of area, throutput and chaotic random outputs. 

In 2013, Lahcene Merah and all, demonstrate by mixing to chaotic map they can increase the security against plaintex attacks, by coupling a chaotic encryption system (ENS) based on 2-D Hénon map used to generate the chaotic sequence, and control system (CRS) based on 1-D logistic map to control a multiplexer to choose the output of ENS according to the value generated by the logistic map by XORing the MSB of 32-bit of CSR with his it neighbor LSB. The results verified a good autocorrelation, sensitivity to initial parameters. However to increase NIST test and to resist more for the attacks, they process all the outputs sequence of the system in to a logic circuit that Xor the output CSR with the output system sequence following some rules.

. 


%%%%%%%%%%%%%%
Non-Linear Chaotic Dynamic
In 2013, \textbf{Hariprasad  NagaDeepa. Ch.  }In this paper, we discuss some aspects of linear and non Linear pseudorandom number generators (PRNG's). To overcome the predictability problem nonlinear chaos-based prngs (CBprngs) were proposed, it is efficient in hardware cost, but due to quantization error there exists short periods in such nonlinear prngs. They produce only one bit per iteration hence throughput rate is low. And then to produce long periods and high throughput rate reseeding-mixing PRNG (RM-PRNG) were proposed. The RM-PRNG consists of a CB-PRNG and MRG. The reseeding method removes the short periods in the CB-PRNG and by mixing MRG with CB-PRNG the overall system period length increases. Vector Mixing Module is implemented by an auxiliary linear generator (ALG) and output construction. By mixing Xt+1 with the output Yt+1 from ALG in Vector Mixing Module,

In 2011, Pawel Dabal, Ryszard Pelka. This paper presents results of studies on the implementation of pseudo-random bit generators based on a nonlinear dynamic chaotic system. Design of chaotic PRNG: The logistic function generator was designed using MATLAB/Simulink with System Generator tool which offers ready to use library of fixed-point arithmetic blocks, that can be directly implemented into the FPGA device. In both cases we need a multiplier and a subtraction block. The multiplier unit was built using DSP48E blocks that can perform 18x25 multiplication.

In 2010, Ziqi Zhu and Hanping Hu. a high efficiency dynamic nonlinear transform arithmetic, which is used to improving the properties of chaosbased PRNG, is designed. Therefore, the DNT module 1 is processing next element of the sequence while the second one is processing the output number of DNT module 1. So the whole system's efficiency is determined by the efficiency of each DNT module. And for each one, the parallel structure guarantees the high efficiency of DNT module.

In 2010, Amit Pande and Joseph Zambreno. In this paper, a chaotic stream cipher is first constructed and then its hardware implementation details using FPGA technology are provided. We present a Modified Logistic Map which has better properties than the Logistic Map - in terms of higher confusion (larger Lyapunov exponent) and a flatter distribution for various parameter values in the bifurcation diagram. The logistic map is a polynomial mapping of degree 2. We present an optimization of usage of DSP multipliers based on above observations for the multiplication of two 64 bit numbers X and Y. By adding two pipelining stages to the 64 $\times$ 64 bits multiplier, we obtained a clock frequency of 93 MHz and required only 16 DSP48E1 slices in the design.

%%%%%%%%%%
Spationtemporal Chaos
In 2009, Yaobin Mao and all. To meet these needs, a spatiotemporal chaotic map is digitized to develop a highly paralleled PRBS generator that accommodates to FPGA (Field Programmable Gate Array) implementation in present paper. in this paper, a spatio-temporal chaos based PRBS generator suitable for FPGA chip implementation is suggested. Compared with traditional chaos based PRBS generators that usually realized serially by software, the proposed generator can simultaneously generate multiple pieces of bit sequences by hardware. The intrinsic operational parallelism of the PRBS generator makes the hardware implementation fairly easy and efficient. Elementary test results show that the throughput of the designed chip can reach high up to 512 Mbits/s under a running condition of 50 MHz clock frequency. Four kinds of mathematical models are usually used to represent spatiotemporal chaotic systems. They are partial differential equation (PDE), coupled ordinary differential equation, coupled mapped lattice (CML) and cellular automata respectively, among which the CML is most widely used due to its appropriate tradeoff between the calculational complexity and the representative of original system. Therefore, a pseudo-random number generator employing CML would achieve high operation speed. Since in above mentioned PRBS generating scheme only integers and some simple arithmetic and logic operations are used, it is easy to be implemented on a chip. 

%%%%%%%%%%%
Fibonacci post-processing
In 2013, Abhinav S. Mansingka. This paper presents a hardware implementation of a robust non-autonomous hyperchaotic-based PRNG driven by a 256-bit LFSR. The original chaotic output is post-processed using a novel technique based on the Fibonacci series, bitwise XOR, rotation, and feedback. This paper considers a new approach to hardware post processing using XOR operation and variable rotation based on Fibonacci series through two feedback loops. chaotic output passing successfully all NIST SP. 800-22 tests. The system is verified on a Xilinx Virtex 4 FPGA achieving throughput up to 13.165 Gbits/s for 16-bit implementations surpassing previously reported CB-PRNGs.  Digital Realization, The proposed 4-D hyperchaotic non-autonomous system is described by the following state space matrix representation of four first order ordinary differential equations. This particular system is digitally implemented in hardware by realizing the numerical solution of the ODE. In this section, a new post processing technique is proposed that uses two rotations and XOR feedback loops: the first loop suppresses short-term predictability using a fixed rotation while the second enhances differential sensitivity using a variable rotation controlled by the Fibonacci series. Feedback Loop 1: Static Rotation Loop 1 implements a rotation and subsequent XOR in feedback. The rotation amount is fixed at 1-bit, axiomatically guaranteeing that the rotation amount and the total input width (64-bits) are relative primes. If loop 2 is neglected, the resulting output after C cycles at the m-th bit position is. Feedback Loop 2: Variable Rotation with Fibonacci Series,  The rotation amount is specified by a Fibonacci series. The native chaotic system suffers from short-term predictability and thus fails the tests. Post processing guarantees full output bus width passes the NIST tests.  Introducing two registers in the post processor segments the combinational logic, the XOR and the variable rotation using Fibonacci series, which reduces the delay introduced by the post processor and preserves the output to be pipelined. While the LFSR and the proposed post processing have added a significant hardware cost.

%%%%%%%%%%%%
FPGA Implementation and Evaluation of Discrete-time Chaotic Generators Circuits

In 2012, Pascal Giard and all. In this paper, implementation of discrete-time chaotic generators widely used in digital communications is studied and evaluated. To our best knowledge, there is no paper which helps us select the optimal chaotic generator based on engineering priorities for a given FPGA. The chosen discrete-time chaotic generators are: 1) Bernoulli map, 2) Chebychev map, 3) Tent map. 





%%% Compare ALL RNG for Statistic Test
\section{Rundom Number Generator:Statistic Test }

\begin{table}[h]
\begin{tabular}{|l|l|l|l|l|}
\hline
RNG  & Algorithm & DieHard & NIST & FIPS   \\ \hline
 	 & LFSR & {[1]} & {[2]} & {[3]}    \\ \cline{2-5} 
PRNG & BBS & {[1]} & {[2]} & {[3]}     \\ \cline{2-5} 
	 & MT & {[1]} & {[2]} & {[3]}     \\ \cline{2-5}
	 & CA & {[1]} & {[2]} & {[3]}     \\ \hline	 
 	 & LFSR & {[1]} & {[2]} & {[3]}     \\ \cline{2-5} 
	 & IRO & {[1]} & {[2]} & {[3]}     \\ \cline{2-5} 
TRNG & FIGARO & {[1]} & {[2]} & {[3]}     \\ \cline{2-5}
	 & STR & {[1]} & {[2]} & {[3]}     \\ \cline{2-5}
	 & MS & {[1]} & {[2]} & {[3]}     \\ \hline	 
  	 & Logistic & {[1]} & {[2]} & {[3]}     \\ \cline{2-5} 
	 & DNT & {[1]} & {[2]} & {[3]}     \\ \cline{2-5} 
Chaotic	 & Spationtemporal & {[1]} & {[2]} & {[3]}     \\ \cline{2-5}
	 & Fibonacci & {[1]} & {[2]} & {[3]}     \\ \cline{2-5}
	 & CI & {[1]} & {[2]} & {[3]}     \\ \hline
\end{tabular}
\end{table}

\textit{DieHard}

\textit{NIST}

\textit{FIPS}

\textit{TestU01}

\textit{AIS}



%%% Compare ALL RNG for Statistic Test
\section{Rundom Number Generator: Cryptography Secure}
\input{5-Cryptography_Secure.tex}

%%% Compare ALL RNG for Statistic Test
\section{Rundom Number Generator: Hardware Implementation}
\input{6-Hardware_Implementation.tex}

%%% Conclusion
\section{Conclusion}
\input{7-Conclusion.tex}

\section{Reference}
\begin{thebibliography}{4}

%%%% CHAOS
\bibitem{lncschap} [CH-1] Pawel Dabal, Ryszard Pelka. : A Chaos-Based Pseudo-Random Bit Generator Implemented in FPGA Device. In: (2011)
\bibitem{lncschap}[CH-2] Pawel Dabal et all.:FPGA Implementation of Chaotic Pseudo-Random Bit Generators. In: (2012)
\bibitem{lncschap}[CH-3] Pawel Dabal, Ryszard Pelka. :A Study on Fast Pipelined Pseudo-Random Number Generator Based on Chaotic Logistic Map. In: 2014)  
\bibitem{lncschap}[CH-4] Lahcene Merah et all, :Coupling Two Chaotic Systems in Order to Increasing the Security of a Communication System - Study and Real Time FPGA Implementation. In (2013) 
\bibitem{lncschap}[CH-5] Hariprasad  NagaDeepa. Ch. :FPGA Implementation of A Cryptography Technology Using Pseudo Random Number Generator. In: (2013)   
\bibitem{lncschap}[CH-6] Ziqi Zhu and Hanping Hu. :A Dynamic Nonlinear Transform Arithmetic for Improving the Properties Chaos-based PRNG*. In (2010) 
\bibitem{lncschap}[CH-7] Yaobin Mao et all. :Design and FPGA Implementation of a Pseudo-Random Bit  Sequence Generator Using Spatiotemporal Chaos. In: (2009)
\bibitem{lncschap}[CH-8] Abhinav S. Mansingka. :Fibonacci-based Hardware Post-Processing for Non-Autonomous Signum Hyperchaotic System. In: (2013)
%%%% TRNG
\bibitem{lncschap}Viktor Fischer and Milos Drutarovsky. [TR-1]  (2002)
\bibitem{lncschap}Martin Simka et al,.:In: (2006) [TR-2] 
\bibitem{lncschap}Martin Simka et al,.:In: (2010) [TR-3]  
\bibitem{lncschap}Martin Simka et al,.:In: (2011) [TR-4] 
\bibitem{lncschap}Michal Varchola et al,.:In: (2008) [TR-5] 
\bibitem{lncschap}Paul Kohlbrenner et Kris Gaj .:In  (2004) [TR-6] 
\bibitem{lncschap}Cristian Klein et al.:In: (2009) [TR-7] ,
\bibitem{lncschap}K.H.Tsoi et al,.:In: (2003) [TR-8] 
\bibitem{lncschap}Markus Dichtl et Jovan Dj.Golic,.:In: (2007) [TR-9] 
\bibitem{lncschap}Abdelkarim Cherkaoui et al,.:In: (2013) [TR-10]
\bibitem{lncschap}Abdelkarim Cherkaoui et al,.:In: (2011) [TR-11]
\bibitem{lncschap}Abdelkarim Cherkaoui et al,.:In: (2013) [TR-12] 
\bibitem{lncschap}Ihor Vasyltsov et al.:In: (2008) [TR-13] 
\bibitem{lncschap}Mehrdad Majzoobi et al.:In: (2011) [TR-14] 
\bibitem{lncschap}Donggeon Lee,.In: (2014) [TR-15]
%%%% CA
\bibitem{lncschap}Tkacik.:In: (2003) [CA-1] 
\bibitem{lncschap}Juan C.Cerda and al.:in: (2012) [CA-2] 
\bibitem{lncschap}Juan C.Cerda and al.:.In: (2012) [CA-3] 
\bibitem{lncschap}Guan et al.:.In: (2004) [CA-4] 
\bibitem{lncschap}Leonidas Kotoulas et al,:.In: (2006) [CA-5] 
\bibitem{lncschap}Petre Anghelescu et al,:.In: (2007) [CA-6] 
\bibitem{lncschap}Ioana Dogaru and Radu Dogaru,:.In: (2010) [CA-7] 
\bibitem{lncschap}Lakshman Raut and David H. K. Hoe,:.In: (2013) [CA-8] 
\bibitem{lncschap}Dogaru Ioana and Dogaru Radu. :. In: (2014) [CA-9] 
  
\end{thebibliography}

\end{document}