\label{Introduction}
Accordingly, for cryptography applications, PRNGs must have the three characteristics. The first of these, the length of the seed used in the RNG should be large enough (say, ρ ≈ 2200 or more), so that an adversary cannot simply run through all possibilities, the Random number generators based on linear recurrences modulo 2 are among the fastest long-period generators currently available but very fast RNGs with huge period length can indeed be constructed this way. The second, the output sequences of random numbers should be statistically indistinguishable from random sequences. The third, random numbers should be unpredictable to an adversary with limited system resources. Should not waste memory (the state should be represented in no more than roughly log2 ρ bits of memory) and allow efficient jumping ahead in order to obtain multiple streams and sub-streams. But these properties do not suffice to imitate independent random numbers.

\noindent Chaos theory is well-studied in mathematics and nature phenomenal that widely exists in nonlinear systems [1, 2, 3]. It has many exceptional good properties such as the sensitivity to initial conditions and parameters, the pseudo-randomness, the topological transitivity, being irregular, non-periodicity, unpredictability, and ability to reciprocal synchronization [4]. However, in practice these solutions have also some important drawbacks and limitations, because the chaotic nature of generated series is non-ideal, due to the limited precision of arithmetic operations and quantization. As a result, instead of random number sequences we get pseudo-random or periodic series. In practice, the chaotic nature of the obtained series is non ideal, due to the finite precision of arithmetic and quantization. As a result we get pseudo-random or periodic series.