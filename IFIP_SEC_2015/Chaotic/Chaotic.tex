%%%%%%%%%%%%%%
\textit{Logistic} Based on earlier study in In 2011 [CH-1] Pawel Dabal and Ryszard Pelka, implement two version of generator using XSG tool based on logistic mapping equation but distribute the equation $X_{t+1} = r*X_{n}*(1-X_{n})$. Then in 2012 [CA-2] propose a studies of HW implementation in FPGA of different chaos system as logistic and Hénon mapping and Frequency depended negative resistor FNDR.  

In 2014 [CH-3] Pawel Dabal and Ryszard Pelka, The authors propose a study of an fast Pipeline PRNG based on chaotic logistic map,and to solve the problem of the short cycle, distortion and correlation, the authors implement two version of PRNG based on simple logistic map equation using pipeline processing to have a a high operation frequency and the ability to generate sequence at different start point. The first is logLUT based on on LUT blocs and high-speed carry line of the FPGA and the second is Log DSP that use directly DSP of FPGA, however they add delays to ensure parallel sequence generation and a complex initial sequence used for a better NIST test results. As results, many configuration and tests based on delays and precision of PRNG applied to have the most combination of area, throutput and chaotic random outputs. 

In 2013 [CH-4] Lahcene Merah and all, demonstrate by mixing to chaotic map they can increase the security against plaintex attacks, by coupling a chaotic encryption system (ENS) based on 2-D Hénon map used to generate the chaotic sequence, and control system (CRS) based on 1-D logistic map to control a multiplexer to choose the output of ENS according to the value generated by the logistic map by XORing the MSB of 32-bit of CSR with his it neighbor LSB. The results verified a good autocorrelation, sensitivity to initial parameters. However to increase NIST test and to resist more for the attacks, they process all the outputs sequence of the system in to a logic circuit that Xor the output CSR with the output system sequence following some rules.

In 2013 [CH-5] Hariprasad and NagaDeepa, the authors demonstrating using the reseeding technique to avoid non-lineair chaotic PRNG as short-period problems, and that by mixing Reseeding module with a non-linear chaotic logistic map (CLM) using a vector mixer module based on auxiliary linear generator ALG. For each fixed point condition by comparing the $X_{t}$ and $X_{t+1}$ sequence of the CLM it increase the reseeding period until it reached, and then pass the result $X_{t+1}$ sequence to vector mixing to generate the final output bu XORing it with the outputs $Y_{t+1}$ of ALG , ${OUT_{t+1}={(X_{t+1}[1:31]) XOR (Y_{t+1}[1:31])}}$.

%%%%%%%%%%%%%%
\textit{Non-Linear Chaotic Dynamic}
Du to degeneration phenomenon and for a high quality of chaotic of PRNG, In 2010 [CH-6] Ziqi Zhu and Hanping Hu present a new high efficiency dynamic nonlinear transform arithmetic DNT. A chaos system based on three DNT process in a parallel structure that transform input 3 times and improve a high cycle-length and distribution output sequence, and that of each DNT module initiate his 2x256 code book and obtain the binary input sequence, then transformed and look up it using inputs as parameters $B_{i+1}= Bi(C(w)R(q))$. 

%%%%%%%%%%
\textit{Spationtemporal Chaos} In 2009 [CH-7] Yaobin Mao et all, implement a new parallel PRNG based on digitized spatio-temporal chaos map using  coupled mapped lattice (CML) model. To achieve a high operation speed, they first deal with continuous domain with digitazed all operand to be suited for HW implementation by using and modifying a bi-directional coupled chaotic map lattice to a finite integer set. then second and to avoid finite precision chaotic map problem, they compute only the significant bit is subject to output and randomly select initial value for each lattice.  
 
%%%%%%%%%%%
\textit{Fibonacci post-processing} In 2013 [CH-8] Abhinav S. Mansingka et al, present a post processing based on Fibonacci series based on LFSR for non-autonmous signum hyperchaotic PRNG. This paper presents a hardware implementation of a robust non-autonomous 4-D hyperchaotic-based PRNG driven by a 256-bit LFSR. The post processing is based on two loop feedback using as first a fixed 1-bit static rotation to suppress the short-term predictability, then the second is based on  a variable rotation controlled using Fibonacci series of K-bit to enhances differential sensitivity if there is a change at any bit when the other bit propagate during M Cycle, $M=({i: K_{i} < 64, K_{i+1}, iEZ^{+}})$.