%%%%%%%%%%%%%%
\textit{Logistic} Based on earlier study in 2012, Pawel Dabal and all show us a study of HW implementation in FPGA of different chaos system as logistic and Hénon mapping and Frequency depended negative resistor FNDR. Many recent research illustrate an optimization using new methods and algorithms to have a better secure chaotic random and Area/Throutput results.

In 2014 Pawel Dabal and Ryszard Pelka, The authors propose a study of an fast Pipeline PRNG based on chaotic logistic map,and to solve the problem of the short cycle, distortion and correlation, the authors implement two version of PRNG based on simple logistic map equation using pipeline processing to have a a high operation frequency and the ability to generate sequence at different start point. The first is logLUT based on on LUT blocs and high-speed carry line of the FPGA and the second is Log DSP that use directly DSP of FPGA, however they add delays to ensure parallel sequence generation and a complex initial sequence used for a better NIST test results. As results, many configuration and tests based on delays and precision of PRNG applied to have the most combination of area, throutput and chaotic random outputs. 

In 2013, Lahcene Merah and all, demonstrate by mixing to chaotic map they can increase the security against plaintex attacks, by coupling a chaotic encryption system (ENS) based on 2-D Hénon map used to generate the chaotic sequence, and control system (CRS) based on 1-D logistic map to control a multiplexer to choose the output of ENS according to the value generated by the logistic map by XORing the MSB of 32-bit of CSR with his it neighbor LSB. The results verified a good autocorrelation, sensitivity to initial parameters. However to increase NIST test and to resist more for the attacks, they process all the outputs sequence of the system in to a logic circuit that Xor the output CSR with the output system sequence following some rules.

%%%%%%%%%%%%%%
\textit{Non-Linear Chaotic Dynamic} In 2013 Hariprasad and NagaDeepa, the authors demonstrating using the reseeding technique to avoid non-lineair chaotic PRNG as short-period problems, and that by mixing Reseeding module with a non-linear chaotic logistic map (CLM) using a vector mixer module based on auxiliary linear generator ALG. For each fixed point condition by comparing the Xt and Xt+1 sequence of the CLM we increase the reseeding period until it reached, and then pass the result Xt+1 sequence to vector mixing to generate the final output bu XORing it with ALG (Yt+1) OUTt+1=Xt+1[1:31] XOR Yt+1[1:31].

In 2011, Pawel Dabal, Ryszard Pelka. This paper presents results of studies on the implementation of pseudo-random bit generators based on a nonlinear dynamic chaotic system. Design of chaotic PRNG: The logistic function generator was designed using MATLAB/Simulink with System Generator tool which offers ready to use library of fixed-point arithmetic blocks, that can be directly implemented into the FPGA device. In both cases we need a multiplier and a subtraction block. The multiplier unit was built using DSP48E blocks that can perform 18x25 multiplication.

In 2010, Ziqi Zhu and Hanping Hu. a high efficiency dynamic nonlinear transform arithmetic, which is used to improving the properties of chaosbased PRNG, is designed. Therefore, the DNT module 1 is processing next element of the sequence while the second one is processing the output number of DNT module 1. So the whole system's efficiency is determined by the efficiency of each DNT module. And for each one, the parallel structure guarantees the high efficiency of DNT module.


%%%%%%%%%%
\textit{Spationtemporal Chaos} In 2009, Yaobin Mao and all. To meet these needs, a spatiotemporal chaotic map is digitized to develop a highly paralleled PRBS generator that accommodates to FPGA (Field Programmable Gate Array) implementation in present paper. in this paper, a spatio-temporal chaos based PRBS generator suitable for FPGA chip implementation is suggested. Compared with traditional chaos based PRBS generators that usually realized serially by software, the proposed generator can simultaneously generate multiple pieces of bit sequences by hardware. The intrinsic operational parallelism of the PRBS generator makes the hardware implementation fairly easy and efficient. Elementary test results show that the throughput of the designed chip can reach high up to 512 Mbits/s under a running condition of 50 MHz clock frequency. Four kinds of mathematical models are usually used to represent spatiotemporal chaotic systems. They are partial differential equation (PDE), coupled ordinary differential equation, coupled mapped lattice (CML) and cellular automata respectively, among which the CML is most widely used due to its appropriate tradeoff between the calculational complexity and the representative of original system. Therefore, a pseudo-random number generator employing CML would achieve high operation speed. Since in above mentioned PRBS generating scheme only integers and some simple arithmetic and logic operations are used, it is easy to be implemented on a chip. 

%%%%%%%%%%%
\textit{Fibonacci post-processing} In 2013, Abhinav S. Mansingka. This paper presents a hardware implementation of a robust non-autonomous hyperchaotic-based PRNG driven by a 256-bit LFSR. The original chaotic output is post-processed using a novel technique based on the Fibonacci series, bitwise XOR, rotation, and feedback. This paper considers a new approach to hardware post processing using XOR operation and variable rotation based on Fibonacci series through two feedback loops. chaotic output passing successfully all NIST SP. 800-22 tests. The system is verified on a Xilinx Virtex 4 FPGA achieving throughput up to 13.165 Gbits/s for 16-bit implementations surpassing previously reported CB-PRNGs.  Digital Realization, The proposed 4-D hyperchaotic non-autonomous system is described by the following state space matrix representation of four first order ordinary differential equations. This particular system is digitally implemented in hardware by realizing the numerical solution of the ODE. In this section, a new post processing technique is proposed that uses two rotations and XOR feedback loops: the first loop suppresses short-term predictability using a fixed rotation while the second enhances differential sensitivity using a variable rotation controlled by the Fibonacci series. Feedback Loop 1: Static Rotation Loop 1 implements a rotation and subsequent XOR in feedback. The rotation amount is fixed at 1-bit, axiomatically guaranteeing that the rotation amount and the total input width (64-bits) are relative primes. If loop 2 is neglected, the resulting output after C cycles at the m-th bit position is. Feedback Loop 2: Variable Rotation with Fibonacci Series,  The rotation amount is specified by a Fibonacci series. The native chaotic system suffers from short-term predictability and thus fails the tests. Post processing guarantees full output bus width passes the NIST tests.  Introducing two registers in the post processor segments the combinational logic, the XOR and the variable rotation using Fibonacci series, which reduces the delay introduced by the post processor and preserves the output to be pipelined. While the LFSR and the proposed post processing have added a significant hardware cost.

%%%%%%%%%%%%
FPGA Implementation and Evaluation of Discrete-time Chaotic Generators Circuits

In 2012, Pascal Giard and all. In this paper, implementation of discrete-time chaotic generators widely used in digital communications is studied and evaluated. To our best knowledge, there is no paper which helps us select the optimal chaotic generator based on engineering priorities for a given FPGA. The chosen discrete-time chaotic generators are: 1) Bernoulli map, 2) Chebychev map, 3) Tent map. 



