
In 2014, Pawel Dabal, Ryszard Pelka. In this paper we propose a novel design and FPGA implementation of high-speed pseudo-random number generator (PRNG) based on the pipelined processing and chaotic logistic map.  The main goal of the project was to obtain the highest possible operating frequency of the PRNG, which has been achieved by using fast, pipelined architecture. An important issue of practical implementation of chaotic maps in digital systems is due to the essential differences of the dynamics observed for discrete and continuous chaotic systems. As a result of discretization one can observe in output sequences short cycles, distortion of data distribution and correlations. These difficulties are important and serious challenge especially when designing cryptographic solutions based on chaos theory. All variants of implemented chaotic maps use elementary arithmetic operations: addition, subtraction, multiplication and division. We set the parameter r = 4, so that the multiplication by factor r can be calculated by fast 2-bit shift of an argument to the left. We have prepared two basic configurations of the PRNG. We examined two variants of this model. The first, denoted as LogLUT, performs subtraction using combinatorial logic (LUT blocks and high-speed carry lines). The second one, denoted as LogDSP, uses DSP blocks with hardware adders. . This disadvantage, however, can be utilized to generate a number of sequences simultaneously for different starting points. However, more complex multipliers needed by PRNGs operating at higher precision and with longer delays require rapidly growing amounts of FPGA resources. This demand rises almost 4.5 times for the number of LUTs and 15 times for the number of flip-flops when pDelayS = 3 and pDelayM = 12. According to our knowledge the presented design of PRNG is the first approach to implementation of chaotic generator in Zynq FPGA reported so far. 
%%%%%%%%%%%

In 2013, Abhinav S. Mansingka. This paper presents a hardware implementation of a robust non-autonomous hyperchaotic-based PRNG driven by a 256-bit LFSR. The original chaotic output is post-processed using a novel technique based on the Fibonacci series, bitwise XOR, rotation, and feedback. This paper considers a new approach to hardware post processing using XOR operation and variable rotation based on Fibonacci series through two feedback loops. chaotic output passing successfully all NIST SP. 800-22 tests. The system is verified on a Xilinx Virtex 4 FPGA achieving throughput up to 13.165 Gbits/s for 16-bit implementations surpassing previously reported CB-PRNGs.  Digital Realization, The proposed 4-D hyperchaotic non-autonomous system is described by the following state space matrix representation of four first order ordinary differential equations. This particular system is digitally implemented in hardware by realizing the numerical solution of the ODE. In this section, a new post processing technique is proposed that uses two rotations and XOR feedback loops: the first loop suppresses short-term predictability using a fixed rotation while the second enhances differential sensitivity using a variable rotation controlled by the Fibonacci series. Feedback Loop 1: Static Rotation Loop 1 implements a rotation and subsequent XOR in feedback. The rotation amount is fixed at 1-bit, axiomatically guaranteeing that the rotation amount and the total input width (64-bits) are relative primes. If loop 2 is neglected, the resulting output after C cycles at the m-th bit position is. Feedback Loop 2: Variable Rotation with Fibonacci Series,  The rotation amount is specified by a Fibonacci series. The native chaotic system suffers from short-term predictability and thus fails the tests. Post processing guarantees full output bus width passes the NIST tests.  Introducing two registers in the post processor segments the combinational logic, the XOR and the variable rotation using Fibonacci series, which reduces the delay introduced by the post processor and preserves the output to be pipelined. While the LFSR and the proposed post processing have added a significant hardware cost.
%%%%

In 2013, Lahcene Merah and all, The main goal of this paper is to increase the security of a chaotic crypto system especially against known plaintext and chosen plaintext attacks by coupling two chaotic systems, the Hénon map used to generate the chaotic sequence and the logistic map to control a multiplexer that outputs is x(k) or x(k-1) of the Hénon system according to the value generated by the logistic map. In order to remedy this problem we propose a new approach that based on coupling two chaotic systems, the first called control system (CRS) and the second called the encryption system (ENS). The CRS based on the one dimensional chaotic system of the logistic map and the ENS based on the two dimensional chaotic system of the Hénon map. If it is a chaotic system, then the trajectories will vary wildly as you shift the initial conditions even slightly [11], this is well property of chaos, which is to say that the very similar initial conditions can generate two completely different sequences after a number of iterations. Now various methods for chaos-based secure transmission of private information signals have been proposed, some popular methods are additive masking, chaotic switching, chaotic parameter modulation, chaos shift keying and chaotic frequency modulation [14].  \includegraphics*[width=2.47in, height=1.46in, keepaspectratio=false]{image9}Throughput = output word length $\times$ f  = 32 bits $\times$ 25.456 MHz = 814 Mbps\textbf{\textit{\underbar{}}}

In 2013, \textbf{Hariprasad  NagaDeepa. Ch.  }In this paper, we discuss some aspects of linear and non Linear pseudorandom number generators (PRNG's). To overcome the predictability problem nonlinear chaos-based prngs (CBprngs) were proposed, it is efficient in hardware cost, but due to quantization error there exists short periods in such nonlinear prngs. They produce only one bit per iteration hence throughput rate is low. And then to produce long periods and high throughput rate reseeding-mixing PRNG (RM-PRNG) were proposed. The RM-PRNG consists of a CB-PRNG and MRG. The reseeding method removes the short periods in the CB-PRNG and by mixing MRG with CB-PRNG the overall system period length increases. Vector Mixing Module is implemented by an auxiliary linear generator (ALG) and output construction. By mixing Xt+1 with the output Yt+1 from ALG in Vector Mixing Module, 
%%%%%%%%%

In 2012, Pawel Dabal and all. We present recent results of our studies on the FPGA implementation of pseudo random bit generators (PRBGs) based on a chaotic behavior of nonlinear systems. Design of logisitic PRBG, In order to simplify the design, we set the parameter r = 4, and the multiplication by factor r is calculated by simple 2-bit shift of an argument to the left. Design of Hénon PRBG F, Unlike in logistic generator, we cannot replace the multiplication by simple shift operation. The most significant bit is used for the sign, the following three bits for the integer part, and the rest of the bits for the fractional part. The outputs of the chaotic generator are within intervals of bounded maximums and minimums. Therefore, fixed-point numbers representation is an excellent solution for the system realization. 
%%%%%%%
FPGA Implementation and Evaluation of Discrete-time Chaotic Generators Circuits

In 2012, Pascal Giard and all. In this paper, implementation of discrete-time chaotic generators widely used in digital communications is studied and evaluated. To our best knowledge, there is no paper which helps us select the optimal chaotic generator based on engineering priorities for a given FPGA. The chosen discrete-time chaotic generators are: 1) Bernoulli map, 2) Chebychev map, 3) Tent map. 
%%%%
A Chaos-Based Pseudo-Random Bit Generator Implemented in FPGA Device

In 2011, Pawel Dabal, Ryszard Pelka. This paper presents results of studies on the implementation of pseudo-random bit generators based on a nonlinear dynamic chaotic system. Design of chaotic PRNG: The logistic function generator was designed using MATLAB/Simulink with System Generator tool which offers ready to use library of fixed-point arithmetic blocks, that can be directly implemented into the FPGA device. In both cases we need a multiplier and a subtraction block. The multiplier unit was built using DSP48E blocks that can perform 18x25 multiplication. 
%%%
Design and Hardware Implementation of a Chaotic Encryption Scheme for Real-time Embedded Systems

In 2010, Amit Pande and Joseph Zambreno. In this paper, a chaotic stream cipher is first constructed and then its hardware implementation details using FPGA technology are provided. We present a Modified Logistic Map which has better properties than the Logistic Map - in terms of higher confusion (larger Lyapunov exponent) and a flatter distribution for various parameter values in the bifurcation diagram. The logistic map is a polynomial mapping of degree 2. We present an optimization of usage of DSP multipliers based on above observations for the multiplication of two 64 bit numbers X and Y. By adding two pipelining stages to the 64 $\times$ 64 bits multiplier, we obtained a clock frequency of 93 MHz and required only 16 DSP48E1 slices in the design. 

%%%%%
A Dynamic Nonlinear Transform Arithmetic for Improving the Properties Chaos-based PRNG

In 2010, Ziqi Zhu and Hanping Hu. a high efficiency dynamic nonlinear transform arithmetic, which is used to improving the properties of chaosbased PRNG, is designed. Therefore, the DNT module 1 is processing next element of the sequence while the second one is processing the output number of DNT module 1. So the whole system's efficiency is determined by the efficiency of each DNT module. And for each one, the parallel structure guarantees the high efficiency of DNT module.
%%%%%
Design and FPGA Implementation of a Pseudo-Random Bit Sequence Generator Using Spatiotemporal Chaos

In 2009, Yaobin Mao and all. To meet these needs, a spatiotemporal chaotic map is digitized to develop a highly paralleled PRBS generator that accommodates to FPGA (Field Programmable Gate Array) implementation in present paper. in this paper, a spatio-temporal chaos based PRBS generator suitable for FPGA chip implementation is suggested. Compared with traditional chaos based PRBS generators that usually realized serially by software, the proposed generator can simultaneously generate multiple pieces of bit sequences by hardware. The intrinsic operational parallelism of the PRBS generator makes the hardware implementation fairly easy and efficient. Elementary test results show that the throughput of the designed chip can reach high up to 512 Mbits/s under a running condition of 50 MHz clock frequency. Four kinds of mathematical models are usually used to represent spatiotemporal chaotic systems. They are partial differential equation (PDE), coupled ordinary differential equation, coupled mapped lattice (CML) and cellular automata respectively, among which the CML is most widely used due to its appropriate tradeoff between the calculational complexity and the representative of original system. Therefore, a pseudo-random number generator employing CML would achieve high operation speed. Since in above mentioned PRBS generating scheme only integers and some simple arithmetic and logic operations are used, it is easy to be implemented on a chip. 
