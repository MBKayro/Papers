\textit{Phase-Locked Loop} In 2002 Viktor Fischer and Milos Drutarovsky, propose a analysis about extracting randomness from the jitter of the PLL implemented on Altera FPLD. Their studies is based on detecting the jitter by the sampling of the reference clock signal ($F_{CLK}$) using a correlated signal synthesized in the PLL ($F_{CLG}$) where $F_{CLG}=F_{CLK}(K_{M}/K_{D})$, and the maximum distance between the two clock (CLK,CLG) must be minimum $MAX(\Delta T_{min}) < \sigma_{jit}$. However they confirm in ideal environment condition and without jitter the sampled output or random is deterministic under a period of $T{Q}={K_{D}T_{CLK}}={K_{m}T_{CLG}}$, then they conclude in a real condition $\sigma_{jit} \neq 0 $ the randomness is not deterministic and depending on jitter distribution where the $MAX(\Delta T_{min})= T_{CLK}*GCD(2K_{M},K_{D})/4K_{M}$.  

In 2006 Martin Simka et al, The authors demonstrate by taking 2002 as model that by combined more than one PLL even parallel or series, can increase significantly sensitivity on the jitter $S=F_{CLK}MAX(\Delta T_{min})$ and the output-bit of the generator compared to the use of one PLL. The configuration of multiple PLL are based on input/output length, CVO frequency and MUL/DIV factors ($K_{M}/K_{D}$).In 2011 Martin Simka et al 2010 test the impact of the the change on operation condition environment as temperature of an PLL and illustrate that with low bandwidth of PFF cause a higher number of the critical samples, decreases the output jitter and thus increase the tracking jitter. As in application, in 2008 Michal Varchola et al explore embedded system application of TRNG based PLL to extract randomness from the jitter and propose two version where the slower 40kbps can pass the tests.

%%%%%%
\textit{Inverter ring oxialltor} In 2004 Paul Kohlbrenner et Kris Gaj and In 2009 Cristian Klein et al, propose a TRNG based on two ring oscillators clocked by different clock generated by an internal PLL on FPGA, and with low area implement by only one CLB slice. the authors also extract the jitter of the 2 RO using a simpler and eliminate any correlation between successive bits.

In 2003 K.H.Tsoi et al, propose a Hybrid implementation on FPGA of TRNG based on RO and PRNG based on BBS generators and with high operation frequency of 400Mhz. However they generate an off-chip low frequency based resistor and capacitors and it was notice they implement BBS using ALU structure that satisfies the mathematical model as squaring and modulo operation which will perform the clock cycle of each operation by $(4.5*n^{2} + n)$.

In 2010 Michal Varchola and Milos Drularosky,

\textit{FIGARO ring oxialltor} IN 2007 Markus Dichtl et Jovan Dj.Golic, propose a new approach that can replace RO based on inverters and prove higher randomness using XORE combination between Fibonacci (FIRO) and Galois ring oscillators (GARO)[8]. the main key consists of a number, r, of inverters connected in a cascade together with a number of XOR logic gates forming a feedback in an analogous way where the feedback polynomial form is $f(x) = (1+x)h(x)$ where $h(1)=1$ and the result show withe the new method can achieve a stable state less than classical RO. 

%%%%%
\textit{Self-timed ring STR} In [2013] [2011] [2013] Abdelkarim Cherkaoui et al, the authors propose another alternative more robust to environment (power, temperature) than RO based inverter and based on Self-Timed Ring (STR). The SRT approach consist of a ripple of L stage of FIFO as a ring $(C_{i})_{1}{_{\leq}}{_{i}}{_{\leq}}{_{L}}$ with a phase of $\Delta \varphi = T/2L$, and extract jitter of each oxillator stage using two asynchronous handshaking protocol as even that can be (taken or bubble). However and first, to have the randomness bits, that outputs $(S_{i})_{1}{_{\leq}}{_{i}}{_{\leq}}{_{L}}$ events will be samples using a flip-flop by the main clock and the result will be combined by a XOR operation $\psi = s_{1} \oplus s_{2} \oplus ... \oplus s_{L}$. Secondly, the authors suggest that to avoid the limitation frequency of the STR by the long period delay, the maximum frequency is achieve when the propagation delay (forward and reverse static delay) is near to ring accuracy (N° of token and bubble) and $N_{T}/N_{B} \cong D_{ff}/_D{rr} \simeq 1$.
%%%%
\textit{Metastability}
In 2008 Ihor Vasyltsov et al, the authors present a studies of using Metastability phenomen as a entropy source generated by 5 IRO stage. They claim that by implementing the inverter as loop ring and using a Control Clock Generator to switch the connectivity between the IRO stages flowing two mode (MS, Generation), the output voltage converges to metastability level and stays longer than using bi-stable circuit (Flio-Flop) causing a high entropy. However, the authors wan to estimate the robustness of the system after applying the sampling process in a different process and environment variation modes using CMOS process and FPGA, and they find that it must added another stage for a higher quality output as decreasing the operation rate, applying a Von-Neumann post-processing and influences the loads (RC parasitic) of the last inverter, when it was noted that just post-processing is used in FPGA .

In 2011 Mehrdad Majzoobi et al, 
true random numbers on FPGAs by inducing metastability in bi-stable circuit elements. we propose a novel technique to generate true random numbers on FPGA using the flip-flop metastability as a source of randomness. The introduced TRNG core operates within a closed-loop feedback system that actively monitors the output bit probabilities over windows of bit sequences and generates a proportional feedback signal based on any observed bias in the bit probabilities. The feed back mechanism is made possible by performing fine delay tuning using high precision PDLs with picosecond resolution. The delay tuning ensures that the signals arrive simultaneously at the flip-flop to drive it into a metastable state.

– We introduce an FPGA-based TRNG system that utilizes flip-flop metastability as
the source of randomness.
– A novel feedback mechanism is introduced that performs auto-adjustment on delays
in order to make the metastability condition more likely to happen.
– We demonstrate the use of a PDL to perform fine tuning with a precision of higher
than a fraction of a pico-second.
– Highly accurate delay measurement results for PDL are demonstrated.

Programmable delay lines (PDLs) alter the signal propagation delay in a controlled fashion. The common mechanisms used to change the delay includes incrementally altering the length of the signal propagation path. by letting the signal bounce a few times inside the switch matrices of FPGA. is only possible by dynamic reconfigurability. increments/decrements by only using a single lookup table (LUT).

the resolution process is extremely sensitive to operational conditions and circuit noise.

the probability of the output is $Prob{Out = 1} = Q(\Delta/sigma)$ where $Q(x)= {1/\surd{4\Pi}}{\int{_{x}{^{\infty}}} expr{(-u^{2}/2)du}}$

to obtain completely non-deterministic and unpredictable, our method forces the flipflop into metastability by tuning sampling and signal arrival times so they occur as simultaneously as possible (driving Δ → 0) using the PDLs.

To drive the flip-flop into its metastable state, we use an at-speed monitor-and-control mechanism that establishes a closed loop feedback system. The monitor module keeps track of the output bit probabilities over repeated time intervals.

depending on the outputs propaility, The proportional-integral (PI) controller, decides to add/subtract the delay to/from top/bottom paths to calibrate the delay difference so that it gets closer to zero. 

The delay difference (Δ) is updated/corrected based on the feedback signal.
Ä = Äp + Äb . Äf .

However, due to manufacturing process variability, the δs slightly vary from one PDL to another. 