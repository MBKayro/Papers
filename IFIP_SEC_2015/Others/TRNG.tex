\textit{Phase-Locked Loop PLL}
In 2008 Michal Varchola and al, To demonstrate advantages of this platform a TRNG based on internal PLL was implemented, but TRNGs based on other principles can be also tested. Great advantage of Actel Fusion FPGA is the possibility to create whole system with on-chip analog measurements. The basic principle behind the TRNG shown in Fig. 3 is to extract the randomness from the jitter of the clock signals synthesized in the embedded analog PLLs [2]. The jitter is detected by the sampling of a reference signal CLJ using a rationally related (clock) signal CLK synthesized in the on-chip analog PLLs with frequencies. It was observed that 16-bit data could be transmitted each 2 periods of CopreMP7 system clock by ModelSim simulation.Two configurations of TRNG have been tested operating at following bit-rates - 40 kbps and 1 Mbps. Slower 40 kbps TRNG pass the test very well but faster 1 Mbps does not

In 2011 Martin Simka and al, In the paper we analyze behavior of the Phase-Locked Loop (PLL) based TRNG in changing working environment. The frequency of signals synthesized by PLL may be naturally influenced by chip temperature. We show what impact the temperature has on the quality of generated random sequence of the PLL-based TRNG. When considering PLL as a source of randomness, three parameters have to be taken into account: 1) the size of generated clock jitter; 2) available ranges of frequency dividers n, m and k together with the VCO maximum and minimum frequency; 3) bandwidth of the PLL loop filter. We can conclude that lower bandwidth of the feedback PLL loop in the configuration B causes higher number of the critical samples. This can be explained by the fact that lower bandwidth decreases PLL output jitter and thus increases the tracking jitter. 

In 2004 Paul Kohlbrenner and Kris Gaj, In this paper we extend a technique that uses on-chip jitter and
PLLs to a much larger class of FPGAs that do not contain PLLs. Our design uses only the Configurable Logic Blocks (CLBs) common to all FPGAs,The source of randomness for our TRNG is two ring oscillators. A Phase Locked Loop (PLL) present on Altera FPGAs was used to produce the two clock signals used in this technique. Our proposed design as shown in Figure 2 consists of two independent and identically configured ring oscillators, a sampling circuit, and a control circuit. By design this configuration exactly fit in one Virtex CLB slice. The sampler circuit extracts the jitter contained in the signals from the two ring oscillators. An important secondary benefit of the control circuit is that it prevents any output from the TRNG if the difference between the cycle lengths of the two ring oscillators is too great. it is important to place the two ring oscillators close to each other. 

In 2006 Martin Simka and al, This paper summarizes possible TRNG configurations and relation between PLL and TRNG parameters. based on extraction of the random tracking jitter using two rationally related clocking signals. The basic principle of the true random number generation by an extraction of the tracking jitter included in clock signal CLJ using clock signal CLK is illustrated. There are three options how the PLLs can be configured in the TRNG in dependency on chosen FPGA: with one PLL, with two parallel PLLs and with two (or more) cascaded PLLs. Each configuration permits to achieve different characteristics (defined in [5]) depending on parameters of PLLs, namely maximum input, output and voltage-controlled oscillator (VCO) frequency, multiplication and division factors, etc. and in this way the needed frequency can be synthesized. use of two PLLs in either parallel or serial (cascaded) configuration can increase significantly sensitivity on the jitter and the output bit-rate of the generator. 

%%%%%%
\textit{Ring oxialltor RO}
In 2006 Drie Schellekens and al, a review on Ring oxialltor definition [6][7]

In 2009 Cristian Klein and al, This paper focuses on the design and implementation of a high-quality and high-throughput true-random number generator (TRNG) in FPGA. Our first attempt was to create a TRNG based on [3], which uses one RO to sample the output of the other RO. We appreciated this approach due to the fact that the whole stream before the post-processing phase is random (although it might be biased a little bit). A post-processing phase is required which consists in a resilience function. A single inverter allows us to create ROs which both even and odd number of latches. To make sure that the inverter does not add more delay the inverter and the first latch are mapped to the same CLB. we chose as the resilience function a simple XOR of 2r-bits. increases the output period of the ring oscillators, it also increases the amount of jitter.

In 2003 K.H.Tsoi and al , Two FPGA based implementations of random number generators intended for embedded cryptographic applications are presented. The first is a true random number generator (TRNG) which employs oscillator phase noise, and the second is a bit serial implementation of a Blum Blum Shub (BBS) pseudorandom number generator (PRNG). the TRNG, oscillator phase noise was used. our implementation uses a very high frequency clock (up to 400 MHz) and does not require a scrambler to achieve good random output. There are several factors which affect the randomness of the output [22]. The first situation is that the duty cycle of Fh may not be 50%. 
In this situation, Fr will have unequal probability of being zero or one. the only off-chip components being two resistors and a capacitor for the TRNG low frequency oscillator. PRNG is a bit serial arithmetic logic unit (ALU). The BBS PRNG performs three functions: seed validation, squaring and modulo operations. Seed validation is
performed once only during initialization, and after that, a squaring and modulo operation are performed each iteration to produce Xi, In total, each iteration of the PRNG requires $(4.5*n^{2} + n)$ clock cycles, where n is the size of the modules in bits.

In 2009 KnutWold and Chik How Tan, In this paper, we analyze the TRNG designed by Sunar et al. (2007) based on XOR of the outputs of several oscillator rings. We propose an enhanced TRNG with better randomness characteristics that does not require postprocessing and passes the statistical tests. We have shown by experiment that the frequencies of the equal length oscillator rings in the TRNG are not identical. The difference is due to the placement of the inverters in the FPGA and the resulting routing between the inverters. In this paper, we examine more closely the TRNG based on oscillator rings proposed by Sunar et al. [2]. We show that the TRNG described in [2] is not random without postprocessing our TRNG has no bias and, therefore, no need for complicated postprocessing. The TRNG consists of several equal length oscillator rings connected to an XOR tree. The output from the XOR tree is sampled by a D flip-flop, and the output signal of the D flip-flop is then postprocessed in order to increase the entropy and remove bias from the random signal. The entropy source of the TRNG is the jitter created by each oscillator ring. The main concern of the authors of [10] is that the XOR-tree and the sampling D flip-flop cannot handle the high number of transitions from the oscillator rings. we suggest an enhancement of the TRNG based on the oscillator rings in [2] by adding an extra D flip-flop after each ring. The frequency will increase with the decreasing number of inverters. The tendency is that the bias increases with the increasing number of rings. however equal length oscillator rings will have the same frequency is not true when the dispersion is decreasing with increasing number of inverters. for restart experiment, the first data bits should be omitted in order to have good quality of the random sequence.

IN 2007 Markus Dichtl and Jovan Dj.Golic, The main objective of this work is to evaluate and analyze the amount of true randomness produced by these oscillators. This is achieved by using the restart
approach, which consists in repeating the experiments from identical starting conditions. Fibonacci and Galois ring oscillators [8] (FIRO and GARO, respectively) are both defined as generalizations of a ring oscillator (RO). They consist of a number, r, of inverters connected in a cascade together with a number of XOR logic gates forming a feedback in an analogous way. the feedback polynomial should be chosen to have a form $f(x) = (1+x)h(x)$. To increase randomness and robustness, it is also proposed to use an
XOR combination of a FIRO and a GARO (FIGARO). compare them with a classical RO based inverters, show very clearly that the classical ROs need more than 5 μs until they reach an approximately stable value and the FIROs and GAROs achieve a more or less stable standard deviation of their output voltages already after about 50 ns. 

%%%%%
\textit{self-timed ring STR}

In 2013 Abdelkarim Cherkaoui and al, The proposed true random number generator (TRNG) exploits the jitter of events propagating in a self-timed ring (STR) to generate random bit sequences at a very high bit rate. the stochastic models are not feasible or at least not plausible, because they combine intrinsically pseudo randomness with true randomness. 
In [9], we showed for the first time that self-timed rings (STR) are a highly suitable source of entropy. Based on these observations, in [10] we proposed the first TRNG principle based on STRs. Self-timed rings (STR) are oscillators that can provide events which are evenly spaced in time and distributed over half an oscillation period of one ring stage. Contrary to inverter ring oscillators, several events can propagate simultaneously in STRs thanks to the asynchronous hand- shake protocol. The signals resulting from the STR outputs are synchronized and their mutual position does not change over time. In contrast, the ring oscillator output signals from [7] drift in time and generate pseudo-randomness.

%%%%
metastability: