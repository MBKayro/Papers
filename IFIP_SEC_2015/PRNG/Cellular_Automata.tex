%%%\section{Random Number Generator: Theories and Classification} 
%
\label{Cellular Automata}
FONCTION\\
The CA will evolve in discrete time steps, where the next state of each cell interacts with their immediate neighbours based upon a local rule and updated simultaneously were initialized using six random seeds.. A cellular automata can be viewed as a state machine consisting of an array of cells which hold their current states. where the top row represents the eight [2x2x2 = 2${}^{3}$ = 8] possible binary configurations for a three-cell neighbourhood and a total of [2${}^{8}$ = 256] elementary CA, each of which can be indexed with an 8-bit binary number and the bottom row represents the next state for the cell of interest. While higher dimensions and state values can be used. The algorithm used to compute the next cell state is referred to as the CA local rule which have certain states [S $\in $ \{0,1\}, i = 0,1,.., N -- 1 ]. For example, the rule number of \{ \textit{I${}_{7}$I${}_{6}$I${}_{5}$I${}_{3}$I${}_{2}$I${}_{1}$I${}_{0}$} =00011110\} is 30. If the rule of a CA involves only XOR logic, then it is called a linear rule. Rules involving XNOR logic are referred to as complement rules. A CA with all the cells having linear rules is called a linear CA, whereas a CA having a combination of XOR and XNOR rules is called additive CA. If all the cells obey the same rule, then the CA is said to be a uniform CA, otherwise, it is a hybrid CA. A CA is said to be a null boundary CA if both the left and right neighbour of the leftmost and rightmost terminal cell is connected to logic 0-state. On the fact that the CAs from class III are chaotic dynamical systems and CAs from class II exhibit periodic behaviour (each state lies in some cycles).
%%%%%%%% END introdction

\underline{\textit{1-d Cellular Automatat:}} 4 Papers has been identified.\\
In [4] 2012, Juan C.Cerda, and all, a cryptographically efficient (i.e. passing the DIEHARD and NIST tests) configuration (hybrid 37-bit Linear Feedback Shift Register - LFSR and 16-bit CA) is reported, requiring about 1.37 LUTs/cell. This paper explores the implementation of an efficient hybrid configuration which combines the bit streams from an LFSR and a CA. 

In [5] 2013, Lakshman Raut and David H. K. Hoe, use the stream ciphers. Although they were reported as passing statistical test suites, they are difficult to scale-up for arbitrary numbers of cells and it is not very clear how they can ensure a large key space. This paper explores the combination of LFSRs and CA as the key components of an efficient stream cipher based on A2U2. The main component of the stream cipher is the key stream generator, which can be viewed as a pseudo-random number generator (PRNG). The use of an NFSR instead of an LFSR improves the resistance of the cipher from various forms of cryptanalysis, such as correlation attacks and algebraic attacks [8]. The proposed stream cipher is composed of five different blocks.
 
In [6] 2012, Jonathan M. Comer, and all, The object Figure 4, Figure5, of this paper is to investigate the quality of the random numbers that can be produced using the aforementioned designs while considering the amount of resources required when implemented linear feedback shift register (LFSRs) and cellular automata (CA) on an FPGA. This paper will focus on one-dimensional CA with each state represented by a single bit. Tkacik proposed a random number generator which combines the outputs of a CA with a LFSR [10]. A hybrid 90/150 37 bit CA was combined with a 43-bit LFSR. This maximal length configuration combined 32 bits from the CA and LFSR to produce a maximal length RNG. It was found that the LFSR and CA must be clocked at different frequencies to create a sequence of numbers that can pass all the DIEHARD tests.

In [13] 2006, Leonidas Kotoulas. In this paper a one dimensional (1-d) Cellular Automaton (CA) for pseudorandom number generation and its reconfigurable hardware implementation. The proposed 1-d CA based on the real time clock sequence (analytical time description) can generate high quality random numbers which can pass all the tests. CA were initiated in the early 1950s as a general framework for modelling complex structures capable of self- reproduction and self-repair [3]. 

\underline{\textit{2-d Cellular Auomatat:}} 1 paper for 2-d \\
In [7] 2009, Ding Jun, and all, a rather inefficient PRNG based on the classical CA rule 30 is reported, with a resource allocation of 234 LE for a 32-cell (bit) CA and about 7.3 LE/cell. In the CA, different cell units can use different rules. So the CA is can be expressed by the sequence of the rules used in the cell units, such as (90, 90, 150). Its rule outcomes are encoded in the binary representation.  Rule 30 is of special interest because it is chaotic, in fact, this rule is used as the random number generator used for large integers in Mathematic. Thus, we mainly use the rule 30 to achieve the most cell units, also provide an interface of the update rules to achieve other cells. 

\underline{\textit{Hybrid Cellular Auomatat:}}3 papers for hybrid\\
In [1] 2014, Dogaru Ioana and Dogaru Radu. Introduce Two solutions based on hybrid cellular automata (HCA) for designing cryptographically secure PRNG and Both solutions are based on hybrid cellular automata and proved to be highly efficient in terms of resources (1 LUT/cell, better than most state of the art solutions) and throughput (equal to clock frequency, limited only by the specific FPGA technology). 
They claim that, the lowest complexity among the 4 solutions was reported for the Bernoulli map.

In [11] 2010,  Ioana Dogaru and Radu Dogaru. This paper introduces a methodology based on algebraic normal form representations and software tools to rapidly generate the VHDL code description of elementary hybrid cellular automata with arbitrary parameters In order to have a full benefit of the CA advantages in signal processing, fully parallel hardware implementations are required. Consequently, in this paper we introduce a software tool for rapid-prototyping of hybrid elementary CA in FPGA. 

In [12] 2007, Petre Anghelescu, This paper FIG1, FIG2, FIG4 present a hardware implementation in a FPGA circuit of an efficient encryption algorithm based on Cellular Automata. The proposed encryption system it is realized using a combination of two CA. We use a first CA as a key stream generator, a CA pseudorandom number generator (PSRG) that combines in some way two rules (the rules 90 and 150), to provide the key sequence. The operation of CA can be represented by a state-transition graph. Each node of the transition graph represents one of the possible states of the CA. In this encryption system we use as valid combination only the configurations that generate cycles of length 8. The switches are activated with the help of the first CA, so the first CA is designed to provide encryption rules for the encryption system. In the proposed 8-bit block cipher scheme, the first CA (pseudo-random number generator) must have 9 bites. First bit it is used to generate the signal So and the other 8 bits are used for SI signals (one bit for each cell of the second CA). In concordance with the CA theory, a single basic CA cell was designed. It consists of a D flip-flop and a logic combinational circuit (LCC). The LCC consists of multiplexers and XNORs to implement the rules of CA and control the loading of data and operation of the CA. When the load control signal (LoadData) is "logic 1", data is loaded into D flip-flop. When LoadData is "logic 0", data is run into the cell according with the rules applied to the rule control signals (S1, SO) and the states of neighbourhoods. After an established number of cycles (1 to 7), the data on the Q output of the flip-flop is sent out and new data is loaded in.