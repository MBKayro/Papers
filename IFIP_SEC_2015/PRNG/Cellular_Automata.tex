%%%\section{Random Number Generator: Theories and Classification} 
%

%% Single N-Dimention CA:
In 2004 [9] Guan et al. 

proposed a one dimensional CA where the rule in each cell changes dynamically based upon the states of the cells within a new neighborhood of three cells . Dubbed “Self-Programmable Cellular Automata” (SPCA), the rules are switched between 90 and 165 or 150 and 105. These rules were selected because they can be easily implemented with XOR gates. Certain combinations of rules and neighborhoods were shown to produce maximal length sequences with good quality random numbers.

In 2006 Leonidas Kotoulas and al, The proposed 1-d CA is based on the real time clock sequence and used for stream cipher. The authors show that by using rules based computer times sequence as year, months to seconds can generate initial state and the length of CA cells. the initial state configuration and simultaneously the length of CA cells the product of all the above numbers, namely day, month, year, hour, min and seconds was calculated. The execution time was decided to be t = x(60 - x). The first rule arises from the product of minutes by seconds. The second rule is the number of minutes divided by the number of seconds multiplied by a constant.

In 2009, Ding Jun and al, 
an inefficient PRNG based on the classical CA rule 30 is reported, because it is chaotic. Thus, we mainly use the rule 30 to achieve the most cell units, also provide an interface of the update rules to achieve other cells.
The ca-prng is a CA with 32 cells, implemented as a 32 bit wide register. Each register has separate update logic that looks at the current state of the register and its two nearest neighbors (with wrap around). The total state update latency for all cells is thus one cycle. The actual update of the registers is controlled by external control signals that allow a user to set the register initial pattern (state) and request generation of new pattern. Loading of initial pattern is accomplished by setting the input-pattern-data port to the desired initial pattern and then asserting the load-input-pattern port for one clock cycle. Requesting a new pattern is accomplished by asserting the next-pattern port. After reset the ca-prng will use rule30 as the update rule. Changing the rule is done by assigning the new rule to the update-rule port and then asserting the load-update-rule for one clock cycle. The generated pattern is available as a 32 bit value on the prng-data port. Figure 3 shows input and output ports of the Pseudorandom Sequence Generator.

In 2010 Ioana Dogaru and Radu Dogaru. This paper introduces a methodology based on algebraic normal form representations and software tools to rapidly generate the VHDL code description of elementary hybrid cellular automata with arbitrary parameters. They claim that One problem with a specified initial state xi (0) for all CA cells is X(t+ 1)= (mi) xor [Cell(Xi−1(t), Xi(t), xi+1(t)), ID)] and the output y = Cell(u1, u2, u3, ID) is that for each particular ID one needs to determine a proper logic expression and rewrite the VHDL code accordingly. an alternative representation of
Boolean functions exists and it is called an Algebraic Normal Form (ANF) representation [15]. For the case of any elementary CA (with cells defined as 3-inputs Boolean functions) the ANF representation is: y = [K0 xor K1(U1) xor K2(U2) xor K3(U3) xor K4(U1*U2) xor ... K1(U1*U2*U3)] xor mask if the number of cells is 8 and use the mask to maximize the period of the operation cycle.
In 2010 [11] and [9] R. Drago, HCA101 was presented a comparison between chaotic map based on HCA and logistic map with finite precision implementation, from the perspective of using them as PN (pseudo-random) sequences in communication systems. It was shown that HCA-map is superior in any respect to the logistic map and in addition with the binary synchronization property allowing simplifying the acquisition circuits needed to reconstruct the phase of the PN sequence in the receiver.

%% Combined N-Dimention CA:
In 2003 Tkacik proposed a hardware random number generator implemented on a custom IC which combines the outputs of a CA with an LFSR. A hybrid 90/150 rules an 37 bit CA was combined with a 43-bit LFSR. This maximal length configuration combined 32 bits from the CA and LFSR to produce a maximal length RNG. It was found that the LFSR and CA must be clocked at different frequencies to create a sequence of numbers that can pass all the DIEHARD tests.

In 2007, Petre Anghelescu and al, a hardware implementation in a FPGA circuit of an efficient encryption algorithm based on Cellular Automata. The proposed encryption system it is realized using a combination of two CA. We use a first CA as a key stream generator, a CA pseudorandom number generator (PSRG) that combines in some way two rules (the rules 90 and 150), to provide the key sequence. The block cipher algorithm presented in this
paper is constructing using the second CA from class II, with rules 51, 153, 195. The operation of CA can be represented by a state-transition graph. Each node of the transition graph represents one of the possible states of the CA. In this encryption system we use as valid combination only the configurations that generate cycles of length 8. The switches are activated with the help of the first CA, so the first CA is designed to provide encryption rules for the encryption system. In the proposed 8-bit block cipher scheme, the first CA (pseudo-random number generator) must have 9 bites. First bit it is used to generate the signal So and the other 8 bits are used for SI signals (one bit for each cell of the second CA). In concordance with the CA theory, a single basic CA cell was designed. It consists of a D flip-flop and a logic combinational circuit (LCC). The LCC consists of multiplexers and XNORs to implement the rules of CA and control the loading of data and operation of the CA. When the load control signal (LoadData) is "logic 1", data is loaded into D flip-flop. When LoadData is "logic 0", data is run into the cell according with the rules applied to the rule control signals (S1, SO) and the states of neighbourhoods. After an established number of cycles (1 to 7), the data on the Q output of the flip-flop is sent out and new data is loaded in.

In 2012 Juan C.Cerda and al, This paper explores the implementation of an efficient hybrid configuration which combines the bit streams from an LFSR and a CA.  The last bit of the LFSR and CA with rules 90/150 are combined in an XOR gate to produce a single random bit per clock cycle. There is a trade off between the degree of site spacing and the throughput of the PRNG. Configurations that can output 1 to 8 bits per clock cycle were explored in this study.

In 2013, Lakshman Raut and David H. K. Hoe, use the stream ciphers. Although they were reported as passing statistical test suites, they are difficult to scale-up for arbitrary numbers of cells and it is not very clear how they can ensure a large key space. This paper explores the combination of LFSRs and CA as the key components of an efficient stream cipher based on A2U2. The main component of the stream cipher is the key stream generator, which can be viewed as a pseudo-random number generator (PRNG). The use of an NFSR instead of an LFSR improves the resistance of the cipher from various forms of cryptanalysis, such as correlation attacks and algebraic attacks [8]. The proposed stream cipher is composed of five different blocks.

In 2014, Dogaru Ioana and Dogaru Radu. Introduce Two solutions based on hybrid cellular automata (HCA) for designing cryptographically secure PRNG and Both solutions are based on hybrid cellular automata and proved to be highly efficient in terms of resources. PRNG-A corresponds to the basic HCA model with a large number of bits (63 in this case, easily extended to any arbitrary number of bits). For such large number of cells masks can be arbitrary chosen still ensuring good cryptographic properties. PRNG-B implements a chain of two HCAs and one possible choice is to have 31 cells per each HCA.