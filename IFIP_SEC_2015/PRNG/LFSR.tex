
The initial value of the LFSR is called the seed, and by changing the seed we can change the sequence and because the operation of the register is deterministic, the stream of values produced by the register is completely determined by its current (or previous) state. Likewise, because the register has a finite number of possible states, it must eventually enter a repeating cycle. However, an LFSR with a well-chosen feedback function can produce a sequence of bits which appears random and which has a very long cycle. That feedback function is called a maximal length polynomial. The bits in the LFSR state which influence the input are called taps. This is called the feedback polynomial or characteristic polynomial. it is known that an LFSR with more taps produces a better sequence of random numbers. However, on an FPGA, adding more taps minimizes the number of LUT-based shift registers that can be utilized. For example, in 4 bit LFSR if the taps are at the 4th and 3rd bits (as shown), then the feedback polynomial is x4 +x3 +1. 

A nonlinear feedback shift registers NFSRs must be included in a key stream generator design to remove the linearity in the encrypted